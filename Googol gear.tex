 

VẠN VẬT NHƯ BIỂU HIỆN CỦA NĂNG LƯỢNG NÉN
Một mô hình Googol Gear cho cấu trúc hiện thực của vũ trụ
Tác giả: Phan Thành Trung
Lĩnh vực: Vật lý nền tảng – Vũ trụ học lý thuyết – Khoa học cấu trúc
Từ khóa: năng lượng duy trì, nén–khóa, Googol Gear, khả thể–hiện thực, bất khả đảo
________________________________________
Tóm tắt
Bài báo này đề xuất một khung lý thuyết tối giản trong đó toàn bộ các cấu trúc tồn tại trong vũ trụ — từ hạt cơ bản, sự sống cho tới ý thức — được hiểu như các biểu hiện khác cấp của một đại lượng duy nhất: năng lượng đầu vào (p-in), khi năng lượng này bị nén và khóa qua một cấu trúc đa tầng kiểu Googol Gear.
Trong khung này, vũ trụ không được hiểu như một hệ “tiến hóa theo thời gian” thuần túy, mà như một hệ đang được duy trì, nơi hiện thực xuất hiện không phải do lựa chọn hay đo đạc, mà do chỉ những cấu hình chịu được nén năng lượng mới tồn tại.

Mô hình này cung cấp một diễn giải thống nhất cho tính bất khả đảo, mũi tên thời gian, sự bền vững của vật chất, sự mong manh của sự sống và sự phụ thuộc năng lượng của ý thức, mà không cần thêm thực thể hay giả thuyết ngoại lai.
________________________________________
1. Giới thiệu
Một câu hỏi nền tảng xuyên suốt vật lý hiện đại là:
vì sao vũ trụ có hình dạng cụ thể như ta quan sát, thay vì ở trong vô hạn khả thể khác?
Các lý thuyết hiện hành thường trả lời câu hỏi này bằng:
•	xác suất,
•	đối xứng bị phá vỡ,
•	phép đo,
•	hoặc điều kiện ban đầu.
Tuy nhiên, những cách tiếp cận đó chủ yếu mô tả kết quả, chứ chưa làm rõ cơ chế cấu trúc khiến một hiện thực được duy trì lâu dài, trong khi phần lớn khả thể không bao giờ xuất hiện.
Bài báo này tiếp cận vấn đề từ một hướng khác:
hiện thực không phải là thứ được chọn, mà là thứ được giữ lại bằng năng lượng.
________________________________________
2. Năng lượng đầu vào (p-in) như đại lượng nền
Ta định nghĩa (p-in) là:
năng lượng đầu vào cần thiết để duy trì sự tồn tại ổn định của bất kỳ cấu trúc nào trong vũ trụ.
Khác với:
•	năng lượng động,
•	năng lượng sinh công,
•	hay năng lượng kích thích,
(p-in) là năng lượng duy trì:
•	duy trì hình dạng,
•	duy trì ranh giới,
•	duy trì chênh lệch giữa “tồn tại” và “tan rã”.
Mệnh đề cơ bản của mô hình là:
pin=0    ⇒    khoˆng coˊ hiện thực ổn địnhp_{in} = 0 \;\;\Rightarrow\;\; \text{không có hiện thực ổn định}

Tồn tại không phải trạng thái tự nhiên, mà là một trạng thái phải được cung cấp liên tục
________________________________________
3. Hai miền: khả thể và hiện thực
3.1 Miền khả thể (ψ-e)
Miền ψ-e bao gồm mọi cấu hình:
•	được phép về mặt vật lý,
•	nhưng chưa bị ràng buộc thành hình dạng bền.
Đặc trưng của ψ-e:
•	nhiều bậc tự do,
•	chi phí năng lượng thấp,
•	không có lịch sử,
•	không có mũi tên thời gian.
ψ-e không phải “hư vô”, mà là biển khả thể chưa bị nén.
________________________________________
3.2 Miền hiện thực (e-)
Miền e- bao gồm các cấu hình:
•	có hình dạng cụ thể,
•	tồn tại trong thời gian,
•	có tính bất khả đảo.
Đặc trưng của e-:
•	bậc tự do thấp,
•	chi phí duy trì cao,
•	phụ thuộc trực tiếp vào (p-in).
________________________________________
4. Googol Gear như cấu trúc nén–khóa
4.1 Định nghĩa
Googol Gear trong bài báo này không phải là một thiết bị cơ học, mà là một cấu trúc trừu tượng gồm nhiều tầng nén, với các đặc điểm:
1.	Mỗi tầng làm giảm bậc tự do của hệ
2.	Mỗi tầng làm tăng chi phí năng lượng để duy trì cấu hình
3.	Toàn bộ chuỗi tạo ra tính bất khả đảo cục bộ
Ta biểu diễn tổng quát:
G:(ψ-e, pin)⟶e-\mathcal{G} : (\psi\text{-e},\, p_{in}) \longrightarrow e\text{-}
________________________________________
4.2 Nén thay cho lựa chọn
Trong mô hình này:
•	không có “lựa chọn trạng thái”,
•	không có “sụp đổ do quan sát”.
Thay vào đó:
mọi cấu hình không chịu được nén năng lượng đều tự động bị loại bỏ.
Hiện thực là phần còn lại sau nén.
________________________________________
5. Vạn vật như các mức nén khác nhau của (p-in)
Trong khung này, sự đa dạng của vạn vật không đến từ bản chất khác nhau, mà từ mức độ nén khác nhau của cùng một đại lượng (p-in).
•	Hạt cơ bản: mức nén cao, bền vững, ít bậc tự do
•	Nguyên tử – phân tử: nén vừa, có cấu trúc nhưng linh hoạt
•	Sự sống: nén động, yêu cầu dòng (p-in) liên tục
•	Ý thức: nén tinh vi, cực kỳ nhạy với biến thiên năng lượng
•	Vũ trụ: nén nền, giữ cấu trúc không–thời gian
Vạn vật không phải “khác nhau”, mà là “khác mức nén”.
________________________________________
6. Bất khả đảo và mũi tên thời gian
Một hệ quả trực tiếp của Googol Gear là:
•	năng lượng dùng để duy trì cấu trúc không thể hoàn trả,
•	không tồn tại phép biến đổi cục bộ đưa e- quay lại ψ-e.
Do đó:
G−1    khoˆng toˆˋn tại trong mieˆˋn hiện thực
Mũi tên thời gian xuất hiện không do thống kê,
mà do bất đối xứng cấu trúc của quá trình nén.
________________________________________
7. Big Bang và kết thúc vũ trụ dưới góc nhìn nén
7.1 Big Bang
Big Bang được hiểu như:
•	sự kiện kích hoạt dòng (p-in)
•	thời điểm Googol Gear bắt đầu vận hành.
Không phải “khởi đầu của mọi thứ”, mà là khởi đầu của hiện thực bền.
________________________________________
7.2 Kết thúc vũ trụ
Khi (p-in) không còn đủ để duy trì nén–khóa:
•	hiện thực không “tắt dần”,
•	mà mất ổn định cấu trúc và tái mã hóa.
Không hủy diệt,
chỉ chuyển khung tồn tại.
________________________________________
8. Thảo luận
Mô hình Googol Gear:
•	không phủ nhận các lý thuyết hiện hành,
•	nhưng cung cấp một nền cấu trúc cho chúng.
Các cơ chế như:
•	đo đạc,
•	tối ưu toàn cục,
•	hội tụ xác suất,
được hiểu lại như biểu hiện toán học của một quá trình nén năng lượng sâu hơn.
________________________________________
9. Kết luận
Bài báo này đề xuất rằng:
Toàn bộ vạn vật trong vũ trụ là những biểu hiện khác cấp của cùng một đại lượng — năng lượng đầu vào (p-in) — khi bị nén và khóa qua một cấu trúc Googol Gear.
Hiện thực không được chọn.
Hiện thực được giữ lại bằng năng lượng.
Đó là cái giá của tồn tại.
________________________________________
Ghi chú:
Mô hình này không nhằm thay thế các lý thuyết vật lý cụ thể, mà đóng vai trò như một khung nền, giúp thống nhất cách hiểu về tồn tại, bất khả đảo và sự bền vững của cấu trúc trên mọi thang đo.
PHỤ LỤC A — TOÁN TỬ CHO MÔ HÌNH NÉN–KHÓA GOOGOL
A.1. Không gian trạng thái và ký hiệu
Ta xét một không gian trạng thái tổng quát H\mathcal{H} (có thể hiểu là không gian Hilbert, hoặc không gian trạng thái mở rộng cho hệ phức tạp).
Chia H\mathcal{H} thành hai miền:
•	Miền khả thể (ψ-e): Hψ⊆H\mathcal{H}_{\psi}\subseteq \mathcal{H}
•	Miền hiện thực (e-): He⊆H\mathcal{H}_{e}\subseteq \mathcal{H}
Một trạng thái trong H\mathcal{H} ký hiệu là xx (hoặc ∣x⟩|x\rangle nếu cần dạng lượng tử).
________________________________________
A.2. Hàm “chi phí tồn tại” và năng lượng đầu vào (p-in)
Đặt một hàm chi phí duy trì (hay “giá tồn tại”):
C:H→R≥0\mathcal{C}:\mathcal{H}\rightarrow \mathbb{R}_{\ge 0}
C(x)\mathcal{C}(x)biểu diễn mức năng lượng tối thiểu cần cấp liên tục để trạng thái xxx tồn tại bền trong e-.
Năng lượng đầu vào được ký hiệu pin≥0p_{in}\ge 0.
Điều kiện khả tồn (existence feasibility):
x∈He⟺C(x)≤pin
Diễn giải:
•	Nếu (p-in) không đủ trả chi phí, cấu hình không tồn tại bền và bị đẩy về miền khả thể.
________________________________________
A.3. Toán tử Nén (Compression Operator)
Ta mô hình hóa “Googol Gear” như chuỗi NN tầng nén.
Mỗi tầng là một ánh xạ:
Ki:H→H,i=1,…,N
Ki\mathcal{K}_i làm giảm bậc tự do hiệu dụng của trạng thái.
Ta dùng một đại lượng đo “độ tự do” D(⋅)\mathcal{D}(\cdot)(có thể là số chiều hiệu dụng, entropy thông tin, hoặc độ phức tạp mô tả):
D(Ki(x))≤D(x)
Toán tử nén tổng:
K(N)≡KN∘⋯∘K2∘K1\mathcal{K}^{(N)} \equiv \mathcal{K}_N\circ \cdots \circ \mathcal{K}_2\circ \mathcal{K}_1
________________________________________
A.4. Toán tử Khóa (Lock Operator) và tính bất khả đảo cục bộ
Đặt một toán tử khóa L\mathcal{L} sao cho sau khi nén đủ sâu, hệ rơi vào miền “không có đường lui cục bộ”.
L:H→H
Tính chất then chốt:
L−1 khoˆng toˆˋn tại như một pheˊp bieˆˊn đổi lieˆn tục trong mieˆˋn hiện thực
Hoặc phát biểu theo kiểu “cục bộ”:
Với một nhiễu nhỏ δ\delta quanh trạng thái đã khóa xL=L(x)x_L = \mathcal{L}(x), không tồn tại ánh xạ cục bộ FF sao cho:
F(xL+δ)≈xvới mọi δ đủ nhỏ
Diễn giải:
•	Ta có thể “đập vỡ” cấu trúc,
•	nhưng không thể “tháo ngược” để khôi phục khả thể ban đầu.
________________________________________
A.5. Toán tử Nén–Khóa Googol tổng quát G\mathcal{G}
Gộp nén và khóa, ta có:
G(N)≡L∘K(N)
Đây là “Googol Gear” ở dạng toán tử.
Mô hình chuyển miền:
(Hψ,pin)→ G(N) He
nhưng chỉ với điều kiện khả tồn:
C(G(N)(x))≤pin
________________________________________
A.6. Toán tử Lọc–Tồn tại (Existence Filter)
Ta định nghĩa một toán tử lọc phụ thuộc (p-in):
Fpin:H→H\mathcal{F}_{p_{in}}:\mathcal{H}\rightarrow \mathcal{H}
Fpin(x)={x,neˆˊu C(x)≤pin∅,neˆˊu C(x)>pin\mathcal{F}_{p_{in}}(x)= \begin{cases} x, & \text{nếu } \mathcal{C}(x)\le p_{in}\\ \varnothing, & \text{nếu } \mathcal{C}(x)>p_{in} \end{cases}
Ý nghĩa:
•	∅\varnothing là ký hiệu “không được hiện thực hóa”.
Khi đó quá trình hiện thực hóa hoàn chỉnh có thể viết:
e-=Fpin(G(N)(ψ-e))e\text{-} = \mathcal{F}_{p_{in}}\big(\mathcal{G}^{(N)}(\psi\text{-e})\big)
________________________________________
A.7. Dạng “tối ưu” tương thích Tuy’s cut 
Để liên hệ với “tối ưu toàn cục” mà vẫn giữ cơ chế nén–khóa, ta đặt một hàm mục tiêu:
Φ(x)≡C(x)+λ D(x)
Trong đó:
•	C(x)\mathcal{C}(x): chi phí duy trì (năng lượng)
•	D(x)\mathcal{D}(x): độ tự do/độ phức tạp
•	λ>0\lambda>0: trọng số cấu trúc
Khi đó, trạng thái hiện thực hóa có thể hiểu như nghiệm của bài toán:
x\*=arg⁡min⁡x∈HΦ(x)với raˋng buộcC(x)≤pin
Điểm khác biệt so với “sụp đổ do đo đạc”:
•	“chọn nghiệm” không do quan sát,
•	mà do ràng buộc năng lượng duy trì và nén bậc tự do.
________________________________________
A.8. Toán tử Rò (Leak Operator) và suy sụp cấu trúc
Đặt một toán tử rò năng lượng / mất ổn định:
E:He→R≥0\mathcal{E}: \mathcal{H}_e \rightarrow \mathbb{R}_{\ge 0}
E(x)\mathcal{E}(x) biểu diễn tốc độ tiêu tán hoặc rò rỉ làm tăng chi phí duy trì hiệu dụng.
Ta định nghĩa ngưỡng suy sụp:
C(x)+E(x)>pin⇒x∉He
Diễn giải:
•	khi rò rỉ vượt khả năng bù (p-in), trạng thái không còn tồn tại bền.
________________________________________
A.9. Toán tử Tái mã hóa (Reset Operator)
Khi hệ không thể duy trì e-, nó không “quay về” ψ-e theo đường nghịch đảo, mà tái mã hóa:
R:He→Hψ′
Trong đó Hψ′\mathcal{H}'_{\psi} là miền khả thể mới (tái chuẩn hóa).
Tính chất:
•	R\mathcal{R} không là nghịch đảo của G(N)\mathcal{G}^{(N)}
•	R\mathcal{R} là một đổi khung (thay đổi cấu trúc khả thể)
________________________________________
A.10. Ba mệnh đề dạng định lý
Định lý 1 — Tồn tại đòi hỏi năng lượng duy trì
Nếu C(x)>pin\mathcal{C}(x)>p_{in}thì xx không thể thuộc He\mathcal{H}_e.
Định lý 2 — Bất khả đảo cục bộ
Nếu xL=L(x)x_L=\mathcal{L}(x) là trạng thái đã khóa, thì không tồn tại ánh xạ liên tục cục bộ FF sao cho F(xL)≈xF(x_L)\approx x
Định lý 3 — Mũi tên thời gian là hệ quả của khóa
Nếu G(N)=L∘K(N)\mathcal{G}^{(N)}=\mathcal{L}\circ \mathcal{K}^{(N)} không tồn tại cục bộ, thì tiến trình ψ-e→e-\psi\text{-e}\rightarrow e\text{-} là bất đối xứng thuận–nghịch và tạo ra một hướng ưu tiên (mũi tên thời gian) ở cấp cấu trúc.
________________________________________
A.11. Ghi chú
•	D(x)\mathcal{D}(x): entropy, độ phức tạp mô tả, số chiều hiệu dụng
•	C(x)\mathcal{C}(x): năng lượng tự do tối thiểu cần để duy trì cấu hình
•	E(x)\mathcal{E}(x): tốc độ tiêu tán (rò, nhiễu, mất ổn định)
PHỤ LỤC B — MÔ HÌNH MÔ PHỎNG TỐI GIẢN (TOY MODEL)
B.1. Mục đích
Phụ lục này xây dựng một mô hình rời rạc đơn giản để minh hoạ các mệnh đề chính:
1.	Hiện thực e− là tập các cấu hình thỏa C(x)≤pin\mathcal{C}(x)\le p_{in}
2.	Chuỗi nén K(N)\mathcal{K}^{(N)} làm giảm “độ tự do” D\mathcal{D} nhưng tăng “giá tồn tại” C\mathcal{C}.
3.	Sau khi “khóa” L\mathcal{L}, quá trình không thể đảo cục bộ.
4.	Khi rò E\mathcal{E} vượt mức bù, hệ sụp và tái mã hóa R\mathcal{R}.
________________________________________
B.2. Không gian trạng thái rời rạc
Xét không gian trạng thái hữu hạn:
H={xm,k∣m∈{0,1,…,M}, k∈{1,2,…,Km}}
Trong đó:
•	mm là mức nén (compression level).
•	kk là chỉ số vi trạng thái trong cùng một mức.
Diễn giải:
•	m=0m=0: miền “khả thể thô” (rộng, nhiều lựa chọn).
•	mm tăng: hệ bị nén mạnh hơn, ít vi trạng thái hơn.
Ta giả định số vi trạng thái giảm theo mmm:
Km+1<Km
Ví dụ minh họa đơn giản:
•	M=5M=5
•	Km=2(M−m)K_m = 2^{(M-m)}
→ mức nén càng cao thì số vi trạng thái càng ít.
________________________________________
B.3. Độ tự do D\mathcal{D} và chi phí tồn tại C\mathcal{C}
B.3.1. Độ tự do D\mathcal{D}
Đặt:
D(xm,k)=log⁡Km\mathcal{D}(x_{m,k})=\log K_m
Ý nghĩa: mức có nhiều vi trạng thái thì “tự do” lớn hơn.
________________________________________
B.3.2. Chi phí tồn tại C\mathcal{C}
Đặt:
C(xm,k)=C0+α m\mathcal{C}(x_{m,k}) = C_0 + \alpha \, m
với C0≥0C_0\ge 0, α>0\alpha>0
Ý nghĩa: càng nén sâu (m lớn), càng đắt để duy trì.
________________________________________
B.4. Năng lượng đầu vào (p-in) và miền hiện thực e−
Định nghĩa miền hiện thực (trong toy model):
He(pin)={x∈H∣C(x)≤pin}\mathcal{H}_e(p_{in})=\{x\in\mathcal{H}\mid \mathcal{C}(x)\le p_{in}\}
Vì C\mathcal{C} tăng theo mmm, tồn tại một mức nén tối đa mà hệ còn “chịu được”:
m\*(pin)=max⁡{m∣C0+αm≤pin}m^\*(p_{in})=\max\{m\mid C_0+\alpha m \le p_{in}\}
Khi đó:
•	Các mức m≤m\*có thể tồn tại bền.
•	Các mức m>m\*bị “rớt khỏi hiện thực”.
________________________________________
B.5. Toán tử nén K\mathcal{K} và chuỗi nén K(N)\mathcal{K}^{(N)}
B.5.1. Nén một tầng
Định nghĩa K\mathcal{K} làm tăng mức nén:
K(xm,k)=xm+1,fm(k)\mathcal{K}(x_{m,k}) = x_{m+1, f_m(k)}
Trong đó fm:{1,…,Km}→{1,…,Km+1}f_m:\{1,\dots,K_m\}\to \{1,\dots,K_{m+1}\} là ánh xạ “gộp” nhiều vi trạng thái vào một vi trạng thái ít hơn.
Ví dụ: nếu Km+1=Km/2K_{m+1}=K_m/2 thì có thể lấy:
fm(k)=⌈k2⌉
Khi đó:
D(K(x))<D(x),C(K(x))>C(x)\mathcal{D}(\mathcal{K}(x)) < \mathcal{D}(x),\quad \mathcal{C}(\mathcal{K}(x)) > \mathcal{C}(x)
B.5.2. Chuỗi nén NN tầng
K(N)=K∘⋯∘K⏟N laˆˋn\mathcal{K}^{(N)} = \underbrace{\mathcal{K}\circ\cdots\circ \mathcal{K}}_{N\ \text{lần}}
________________________________________
B.6. Toán tử khóa L\mathcal{L}: làm mất khả năng “tháo lui”
Trong toy model, ta mô hình hóa “khóa” bằng cách gộp thông tin lịch sử (path history) vào trạng thái, khiến việc đảo ngược không thể thực hiện cục bộ.
Cụ thể, mở rộng không gian trạng thái:
H~=H×Σ\widetilde{\mathcal{H}} = \mathcal{H} \times \Sigma
Trong đó Σ\Sigma là tập “dấu vết” (dấu vết có thể hiểu là chuỗi ký hiệu).
Đặt:
L(xm,k)=(xm,k,σm,k)
với σm,k\sigma_{m,k} là “dấu vết” phụ thuộc đường đi nén và ràng buộc cấu trúc.
Tính chất khóa:
Hai trạng thái khác nhau ở H\mathcal{H} sau nén có thể có dấu vết khác nhau, nhưng từ một trạng thái đã khóa (x,σ)(x,\sigma) thì không thể phục hồi duy nhất đường đi và trạng thái trước đó bằng nhiễu cục bộ lên xx (vì σ không phục hồi được từ thay đổi nhỏ của xx).
Đây là cách rời rạc để hiện thực hóa mệnh đề:
L−1 khoˆng toˆˋn tại cục bộ
________________________________________
B.7. Toán tử nén–khóa Googol trong toy model
G(N)=L∘K(N)
Hiện thực hóa hoàn chỉnh kèm lọc theo (p-in):
e-=Fpin(G(N)(ψ-e))
Trong toy model, ψ-e\psi\text{-e} có thể hiểu là tập các trạng thái ban đầu ở m=0m=0
________________________________________
B.8. Rò E\mathcal{E} và điều kiện sụp
Định nghĩa rò phụ thuộc mức nén:
E(xm,k)=ϵ0+βm\mathcal{E}(x_{m,k})=\epsilon_0 + \beta m
với ϵ0≥0\epsilon_0\ge 0, β≥0\beta\ge 0
Tổng gánh năng lượng:
Chiệu(x)=C(x)+E(x)
Điều kiện sụp:
C(x)+E(x)>pin⇒x∉He\mathcal{C}(x)+\mathcal{E}(x) > p_{in} \Rightarrow x \notin \mathcal{H}_e
Khi mmm tăng, cả C\mathcal{C} và E\mathcal{E} tăng → hệ càng dễ vượt ngưỡng và sụp nếu (p-in) không tăng tương ứng.
________________________________________
B.9. Toán tử tái mã hóa R\mathcal{R} (Reset)
Khi một trạng thái đã khóa không còn thỏa điều kiện khả tồn, mô hình không cho phép “đảo nén” để quay về trạng thái cũ, mà chuyển sang một miền khả thể mới.
Ta định nghĩa:
R:H~→Hψ′\mathcal{R}:\widetilde{\mathcal{H}}\to \mathcal{H}'_{\psi}
Trong toy model, một cách đơn giản là:
R(xm,k,σ)=x0, g(σ)′\mathcal{R}(x_{m,k},\sigma) = x'_{0,\,g(\sigma)}
tức là trạng thái mới quay về mức m=0 của một không gian khả thể mới, nhưng chỉ số vi trạng thái phụ thuộc σ\sigma.
Điều này thể hiện rõ:
•	reset không phải nghịch đảo,
•	reset thay khung (tái chuẩn hóa miền khả thể).
________________________________________
B.10. Ví dụ số
Chọn:
•	M=5M=5 Km=2(5−m)K_m = 2^{(5-m)}
•	C0=1C_0=1, α=2\alpha=2 → C=1+2m\mathcal{C}=1+2m
•	ϵ0=0\epsilon_0=0, β=1\beta=1→ E=m\mathcal{E}=m
•	pin=8p_{in}=8
Ta có:
Chiệu(m)=(1+2m)+m=1+3m\mathcal{C}_{\text{hiệu}}(m)= (1+2m)+m = 1+3m
Điều kiện tồn tại:
1+3m≤8⇒m≤21+3m \le 8 \Rightarrow m \le 2
Kết quả:
•	Hệ chỉ có thể tồn tại bền tới mức nén m=2m=2.
•	Nếu nén thêm (m=3), dù “nén làm giảm độ tự do”, hệ sụp vì vượt ngưỡng năng lượng duy trì.
> Nén không phải càng sâu càng tốt;
nén sâu đòi hỏi (p-in) đủ lớn để trả chi phí tồn tại.
________________________________________
B.11. Kết luận của toy model
Toy model này cho thấy một cách rời rạc và minh bạch rằng:
1.	Hiện thực là tập trạng thái thỏa điều kiện năng lượng duy trì.
2.	Chuỗi nén làm giảm độ tự do nhưng làm tăng chi phí tồn tại.
3.	Khóa làm mất khả năng đảo cục bộ (không có “tháo lui”).
4.	Vượt ngưỡng → sụp → tái mã hóa, chứ không quay về đường cũ.
PHỤ LỤC C — MINH HOẠ TRỰC QUAN CHO MÔ HÌNH NÉN–KHÓA GOOGOL
C.1. Nguyên tắc minh hoạ
Các hình trong phụ lục này không nhằm mô phỏng dữ liệu thực nghiệm, mà nhằm:
•	trực quan hóa cơ chế nén–khóa,
•	làm rõ vai trò của năng lượng đầu vào (p-in)pin,
•	minh họa tính bất khả đảo và tái mã hóa (reset).
Mỗi hình tương ứng trực tiếp với một nhóm toán tử đã định nghĩa trong Phụ lục A và Phụ lục B.
________________________________________
C.2. Hình 1 — Không gian trạng thái và ngưỡng tồn tại
Nội dung hình
Hình biểu diễn không gian trạng thái rời rạc theo mức nén mmm:
•	Trục ngang: mức nén mmm
•	Trục dọc: chi phí tồn tại hiệu dụng
Chiệu(m)=C(m)+E(m)\mathcal{C}_{\text{hiệu}}(m) = \mathcal{C}(m) + \mathcal{E}(m)Chiệu(m)=C(m)+E(m) 
Một đường ngang biểu diễn ngưỡng năng lượng đầu vào (p-in)pin.
Ý nghĩa
•	Các mức mmm nằm dưới đường (p-in)pin: trạng thái tồn tại bền (thuộc e−e-e−)
•	Các mức mmm vượt (p-in)pin: trạng thái không thể duy trì, bị loại khỏi hiện thực
Chú thích hình (caption – dùng nguyên văn)
Hình 1. Phân bố chi phí tồn tại hiệu dụng theo mức nén mmm. Đường ngang biểu diễn năng lượng đầu vào (p-in)pin. Chỉ các trạng thái có Chiệu≤pin\mathcal{C}_{\text{hiệu}} \le p_{in}Chiệu≤pin mới tồn tại bền trong miền hiện thực e−e-e−. Hình minh hoạ trực tiếp điều kiện khả tồn trong Phụ lục A và ví dụ số trong Phụ lục B.
________________________________________
C.3. Hình 2 — Chuỗi nén–khóa Googol và bất khả đảo
Nội dung hình
Hình gồm ba phần nối tiếp:
1.	Miền khả thể (ψ-e):
o	biểu diễn bằng một vùng rộng, nhiều nút (nhiều vi trạng thái)
2.	Chuỗi nén K(N)\mathcal{K}^{(N)}K(N):
o	các tầng nén liên tiếp
o	số nút giảm dần
o	mỗi tầng có mũi tên đi một chiều
3.	Trạng thái đã khóa L\mathcal{L}L:
o	các trạng thái cuối được gắn “dấu vết”
o	không có mũi tên quay lại
Ý nghĩa
•	Nén làm giảm độ tự do D\mathcal{D}D
•	Khóa làm mất khả năng đảo cục bộ
•	Không tồn tại ánh xạ liên tục từ e−e-e− quay về ψ-e
Chú thích hình
Hình 2. Sơ đồ chuỗi nén–khóa Googol. Miền khả thể ψ-e bị nén qua nhiều tầng K\mathcal{K}K, làm giảm số vi trạng thái và tăng chi phí tồn tại. Toán tử khóa L\mathcal{L}L làm mất khả năng đảo cục bộ, dẫn đến tính bất khả đảo của quá trình hiện thực hóa.
________________________________________
C.4. Hình 3 — Rò năng lượng, sụp cấu trúc và tái mã hóa
Nội dung hình
Hình biểu diễn diễn tiến động của một trạng thái đã khóa:
•	Trục thời gian (hoặc tiến trình)
•	Đường biểu diễn:
C(x)+E(x)\mathcal{C}(x) + \mathcal{E}(x)C(x)+E(x) 
Khi đường này vượt qua (p-in)pin, xảy ra:
•	Sụp cấu trúc
•	Tái mã hóa (reset) sang miền khả thể mới Hψ′\mathcal{H}'_{\psi}Hψ′
Ý nghĩa
•	Sụp không phải là “đảo chiều”
•	Reset không phục hồi trạng thái cũ
•	Reset tạo khung khả thể mới
Chú thích hình
Hình 3. Minh hoạ quá trình rò năng lượng và sụp cấu trúc. Khi tổng chi phí duy trì vượt quá (p-in)pin, trạng thái hiện thực không còn tồn tại bền và hệ chuyển sang tái mã hóa R\mathcal{R}R. Reset không phải là nghịch đảo của quá trình nén–khóa, mà là một đổi khung khả thể.
________________________________________

